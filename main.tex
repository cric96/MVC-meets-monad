\documentclass[presentation]{beamer}
\usepackage{notes_style_v3}
\usecolortheme{myct3}
\usepackage[orientation=landscape,size=custom,width=16,height=9.75,scale=0.5,debug]{beamerposter}
\usepackage{multimedia}

\usepackage{tikz}
\usetikzlibrary{arrows,shapes.geometric,positioning,automata}
\usetikzlibrary{trees,decorations.pathmorphing,calc}
\usetikzlibrary{intersections}

\title[MVC Meets Monad]{MVC meets Monads}
\subtitle{A Functional-based design for a classic architectural pattern}

\author[G.Aguzzi]
{\textbf{Gianluca Aguzzi}}

\usepackage[backend=bibtex,style=alphabetic]{biblatex} %,citestyle=authoryear

\usepackage{pifont}
\usepackage{marvosym}
\newcommand{\demoSymbol}{{\Large \ding{249}}} %{{\Large \Faxmachine}} %{\ding{43}}%

\institute[shortinst]{\normalsize 
%\inst{1} 
\textsc{Alma Mater Studiorum}--Universit\`a di Bologna, Cesena, Italy 
}

\bibliography{biblio} 
\definecolor{pbfilling}{HTML}{ffffff}% filling color for the progress bar
%\definecolor{pbfilling}{HTML}{aaaaaa}% background color for the progress bar
\definecolor{pbbackground}{rgb}{0.36, 0.54, 0.66}

\makeatletter
\def\progressbar@progressbar{} % the progress bar
\newcount\progressbar@tmpcounta% auxiliary counter
\newcount\progressbar@tmpcountb% auxiliary counter
\newdimen\progressbar@pbht %progressbar height
\newdimen\progressbar@pbwd %progressbar width
\newdimen\progressbar@tmpdim % auxiliary dimension
\progressbar@pbwd=\paperwidth
\progressbar@pbht=0.5ex
% the progress bar
\def\progressbar@progressbar{%
    \progressbar@tmpcounta= \insertframenumber % max = ?
    \progressbar@tmpcountb=\inserttotalframenumber      
    \progressbar@tmpdim=.5\progressbar@pbwd
    \multiply\progressbar@tmpdim by \progressbar@tmpcounta
    \divide\progressbar@tmpdim by \progressbar@tmpcountb
    \progressbar@tmpdim=2\progressbar@tmpdim
  \begin{tikzpicture}[rounded corners=0,very thin]
    \shade[top color=pbbackground,bottom color=pbbackground,middle color=pbbackground]
      (0pt, 0pt) rectangle ++ (\progressbar@pbwd, \progressbar@pbht);
      \shade[draw=pbfilling,top color=pbfilling,bottom color=pbfilling,middle color=pbfilling] %
        (0pt, 0pt) rectangle ++ (\progressbar@tmpdim, \progressbar@pbht);
  \end{tikzpicture}%
}

\setbeamertemplate{headline}{%
\leavevmode%
  \vbox{
  \hbox{
    \begin{beamercolorbox}[wd=1\paperwidth,ht=0.7ex,sep=0pt,center,dp=0ex]{white}%
    \progressbar@progressbar%
    \end{beamercolorbox}%
  }
  \hbox{%
  \hypersetup{linkcolor=white}
    \begin{beamercolorbox}[wd=1.02\paperwidth,ht=2.5ex,dp=1.125ex]{headline}%
    \insertsubsectionnavigationhorizontal{\paperwidth}{\bfseries\hfill}{\hfill}
    \end{beamercolorbox}%
  }
  }
}

\let\oldcite\cite
%%% additional documents commands
\newcommand{\scafiweb}{{\sc{}ScaFi-Web}}
\newcommand{\scafi}{{\sc{}ScaFi}}
\renewcommand{\cite}[1]{{\color{blue}\oldcite{#1}}}

\begin{document}

\frame[label=coverpage,noframenumbering,plain]{
\titlepage

  \begin{center}
  Talk {@}\\
  \textbf{PPS 2021} \\
  \end{center}
}

\section{It is all about composition \cite{}}
\begin{frame}[fragile]{It is all about composition \cite{}}
\begin{itemize}
\item Humans tend to divide complex problem in multiple pieces
\item Then, solve each piece
\item Finally compose all solution to solve the initial problem
\item Functional programming is a good way to compose solution :)
\end{itemize}

%\imgv{0.3}{collective-comm.png}

\section{Lecture goals}
\end{frame}
\begin{frame}[fragile]{Lecture goals}
\begin{itemize}
\item Show an end-to-end \emph{functional} application
\item Leverage some well-consolidated functional libraries
\item Understand limitations (if any) and the improvements
\end{itemize}
\end{frame}

\section{Target application}
\begin{frame}[fragile]{Target application: Tic Tac Toe}
\imgh{0.4}{tictactoe.png}
\begin{center}
  Repository: \url{https://github.com/cric96/scala-functional-gui}
\end{center}
\end{frame}

\section{OOP Design}
\begin{frame}[fragile]{OOP Design}
\begin{block}{Everything is an object}
  \begin{itemize}
  \item Clean interface
  \item State incapsulated
  \item Side effect as methods call
  \end{itemize}
\end{block}

\end{frame}

\section{Model}
\begin{frame}[fragile]{Model}
\begin{block}{Core logic}
  \begin{itemize}
    \item Main data to describe model entities (e.g. players?)
    \item Main methods to describe the game logic
  \end{itemize}
\end{block}
\begin{lstlisting}[language=Java]
/* two players (X, O)  */
enum Player {
    X, O, None; //Or? Optional?
}
/* a board 3x3 */
interface TicTacToe {
    Player get(int X, int Y);
    TicTacToe (or void??) update(int x, int y, Player p);
    boolean isOver;
    Player getTurn;
}
\end{lstlisting}
\end{frame}

\section{View}
\begin{frame}[fragile]{View}
\begin{block}{Representation and IO boundary}
  \begin{itemize}
    \item Describe what type of data can consume (for rendering porpuse)
    \item Catch how to handle user input
    \item Ideally, View could be totaly decoupled from the model
  \end{itemize}
\end{block}
\begin{lstlisting}[language=Java]

//a la' view model
interface ViewBoard {
    List<String> getRow(int row);
    List<List<String>> getAllBoard();
}
interface View extends ClickCellSource {
    void render(ViewBoard board);
    void winner(String player);
}
\end{lstlisting}
\end{frame}

\begin{frame}[fragile]{View}
\begin{block}{Desing pattern}
  \begin{itemize}
    \item Input handler is usally solved with the Observer pattern (or also called Listener). 
    \item When is possible, during the design process, is important to identify patterns and reuse common solution :D
  \end{itemize}
\end{block}
\begin{lstlisting}[language=Java]
public interface ClickCellSource {
  void attach(Observer observer);
  interface Observer {
      void notify(int X, int Y);
  }
}
\end{lstlisting}
\end{frame}

\section{Controller}
\begin{frame}[fragile]{Controller}
\begin{block}{Goals}
  \begin{itemize}
    \item Coordinate the interation between the View and Model worlds
    \item Handle concurrency
    \item Adapt data 
  \end{itemize}
\end{block}

\begin{lstlisting}[language=Java]
public interface Game extends ClickCellSource.Observer {
  void start();
}

public class TicTacToeGame implements Game {
  private final TicTacToe ticTacToe;
  private final TicTacToeView ticTacToeView;

  public static TicTacToeGame playWith(
      final TicTacToe ticTacToe, 
      final TicTacToeView ticTacToeView) {...}
  ....

\end{lstlisting}
\end{frame}

\section{Putting all togheter}
\begin{frame}[fragile]{Putting all togheter}

\begin{lstlisting}[language=Java]
public static void main(String[] args) {
  final TicTacToeView view = SwingView.createAndShow(800, 600);
  final TicTacToe model = TicTacToeFactory.startX();
  final Game game = TicTacToeGame.playWith(model, view);
  game.start();
}
\end{lstlisting}
\begin{block}{Clean enogh, isn't it?}
  What do you think?
  \begin{itemize}
    \item What do you think?
    \item Try to rethink using "functional" abstractions
    \item (Monads? Functions? Algebric Data Type?)
    \item Let's go to the "functional" side :)
  \end{itemize}
\end{block}
\end{frame}
%%Todo

\section{Libraries}
\begin{frame}[fragile]{Libraries}

\begin{block}{\href{https://typelevel.org/cats/}{Cats}}
  provides abstractions for functional programming in the Scala programming language
\end{block}

\begin{block}{\href{https://monix.io/}{Monix}}
  high-performance Scala / Scala.js library for composing asynchronous, event-based programs
\end{block}

\end{frame}
%%Todo
\section{Task}
\begin{frame}[fragile]{Task}

\begin{block}{Definition}
  Task represents a specification for a possibly lazy or asynchronous computation, which when executed will produce an A as a result, along with possible side-effects.
\end{block}
\begin{lstlisting}[language=Scala]
  trait Task[+A] {
    final def flatMap[B](f: A => Task[B]): Task[B] = ...
    final def map[B](f : A => B): Task[B] = ...
    //some interesting extesions
    def memoize: Task[A] = ...
}
object Task {
    def pure[A](a : A) : Task[A]
    def defer[A](a : Task[A]) : Task[A]
}
\end{lstlisting}
\begin{center}
  What does it refer you to?
\end{center}

\end{frame}

\section{Task}
\begin{frame}[fragile]{A Little task}

\begin{lstlisting}[language=Scala]
val scheduler = monix.execution.Scheduler.Implicits.global

def someComputation(data : Long) : Task[Long] = 
    Task.pure(data * 1000)

def log(value : String) : Task[Unit] = Task { println(value) }

val main = for {
  data <- someComputation(4)
  _ <- log(s"computations ends with value $data")
} yield (data)

main.runToFuture(scheduler)
\end{lstlisting}
\begin{center}
  \href{https://scalafiddle.io/sf/C4Qon6a/1}{Fiddle}
\end{center}

\end{frame}

\section{Conclusion \& Future Work}

\begin{frame}{Conclusion}

%\imgh{0.9}{example-image}
%

\end{frame}

\begin{frame}[allowframebreaks]{Bibliography}
\def\bibfont{\footnotesize}
\printbibliography
\end{frame}

\end{document}


% \sizedrangedcode{\ssmall}{3}{30}{\labdir/code/code_sec1.txt} 
