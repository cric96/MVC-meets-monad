\documentclass[presentation]{beamer}
\usepackage{notes_style_v3}
\usecolortheme{myct3}
\usepackage[orientation=landscape,size=custom,width=16,height=9.75,scale=0.5,debug]{beamerposter}
\usepackage{multimedia}

\usepackage{tikz}
\usetikzlibrary{arrows,shapes.geometric,positioning,automata}
\usetikzlibrary{trees,decorations.pathmorphing,calc}
\usetikzlibrary{intersections}

\title[MVC Meets Monad]{MVC meets Monads}
\subtitle{A Functional-based design for a classic architectural pattern}

\author[G.Aguzzi]
{\textbf{Gianluca Aguzzi}}

\usepackage[backend=bibtex,style=alphabetic]{biblatex} %,citestyle=authoryear

\usepackage{pifont}
\usepackage{marvosym}
\newcommand{\demoSymbol}{{\Large \ding{249}}} %{{\Large \Faxmachine}} %{\ding{43}}%

\institute[shortinst]{\normalsize 
%\inst{1} 
\textsc{Alma Mater Studiorum}--Universit\`a di Bologna, Cesena, Italy 
}

\bibliography{biblio} 

\definecolor{pbfilling}{HTML}{ffffff}% filling color for the progress bar
%\definecolor{pbfilling}{HTML}{aaaaaa}% background color for the progress bar
\definecolor{pbbackground}{rgb}{0.36, 0.54, 0.66}

\makeatletter
\def\progressbar@progressbar{} % the progress bar
\newcount\progressbar@tmpcounta% auxiliary counter
\newcount\progressbar@tmpcountb% auxiliary counter
\newdimen\progressbar@pbht %progressbar height
\newdimen\progressbar@pbwd %progressbar width
\newdimen\progressbar@tmpdim % auxiliary dimension
\progressbar@pbwd=\paperwidth
\progressbar@pbht=0.5ex
% the progress bar
\def\progressbar@progressbar{%
    \progressbar@tmpcounta= \insertframenumber % max = ?
    \progressbar@tmpcountb=\inserttotalframenumber      
    \progressbar@tmpdim=.5\progressbar@pbwd
    \multiply\progressbar@tmpdim by \progressbar@tmpcounta
    \divide\progressbar@tmpdim by \progressbar@tmpcountb
    \progressbar@tmpdim=2\progressbar@tmpdim
  \begin{tikzpicture}[rounded corners=0,very thin]
    \shade[top color=pbbackground,bottom color=pbbackground,middle color=pbbackground]
      (0pt, 0pt) rectangle ++ (\progressbar@pbwd, \progressbar@pbht);
      \shade[draw=pbfilling,top color=pbfilling,bottom color=pbfilling,middle color=pbfilling] %
        (0pt, 0pt) rectangle ++ (\progressbar@tmpdim, \progressbar@pbht);
  \end{tikzpicture}%
}

\setbeamertemplate{headline}{%
\leavevmode%
  \vbox{
  \hbox{
    \begin{beamercolorbox}[wd=1\paperwidth,ht=0.7ex,sep=0pt,center,dp=0ex]{white}%
    \progressbar@progressbar%
    \end{beamercolorbox}%
  }
  \hbox{%
  \hypersetup{linkcolor=white}
    \begin{beamercolorbox}[wd=1.02\paperwidth,ht=2.5ex,dp=1.125ex]{headline}%
    \insertsubsectionnavigationhorizontal{\paperwidth}{\bfseries\hfill}{\hfill}
    \end{beamercolorbox}%
  }
  }
}

\let\oldcite\cite
%%% additional documents commands
\newcommand{\scafiweb}{{\sc{}ScaFi-Web}}
\newcommand{\scafi}{{\sc{}ScaFi}}
\renewcommand{\cite}[1]{{\color{blue}\oldcite{#1}}}

\begin{document}

\frame[label=coverpage,noframenumbering,plain]{
\titlepage

  \begin{center}
  Talk {@}\\
  \textbf{PPS 2021} \\
  \end{center}
}

\section{Introduction}

\begin{frame}{Introduction}
  \begin{block}{Field Based Coordination~\cite{Viroli-et-al:JLAMP-2019}}
    A coordination model that leverages \lenf{computational fields} to orchestrate large scale systems.
  \end{block}
  \begin{block}{Aggregate Programming~\cite{IEEECIOT2015}}
    A \lenf{top-down} and \lenf{global-to-local} approach 
    to the specification of the collective adaptive behaviour of a distributed system.
  
  \end{block}
    


%\imgv{0.3}{collective-comm.png}

\end{frame}
\begin{frame}[fragile]{Aggregate Programming}
\begin{itemize}
  \item \lbl{Benefits}: 
  \iz{
    \item Abstract over technologies and network topologies
    \item Describe behaviour in a functional and compositional manner
    \item Proved to have, in certain circumstances, relevent properties such as \lenf{eventual consistency}~\cite{DBLP:conf/saso/BealVPD16} and \lenf{self-stabilisation}~\cite{DBLP:conf/saso/BealV14a}
    \item Used in different scenario such as crowd management~\cite{IEEECIOT2015}, UAVs~\cite{DBLP:conf/coordination/CasadeiVAPD19} and Smart cities~\cite{casadei2019scc}
  }
  \item \lbl{Problem:} Hard to use
  \iz{
    \item High learning curve
    \item Plethora of tools involved 
    \item Need of simulations
    \item Lack of tools for real-system monitoring
  }
\end{itemize}

\end{frame}

\begin{frame}[fragile]{Starting point and inspiration}

\begin{block}{Tool-chain}
  \begin{itemize}
  \item \lenf{ScaFi}~\cite{DBLP:conf/ecoop/CasadeiV16}: a modular Scala-based toolchain for aggregate programming comprising an inner-DSL, a simulator and actor-based platfrom
  \item \lenf{Protelis}~\cite{ProtelisSAC14}: a Java-interoperable stand-alone DSL
  \item \lenf{FCPP}~\cite{fcpp}: a lightweight native implementation
  \end{itemize}
\end{block}
  
\begin{block}{Playgrounds}

\begin{itemize}
\item \lenf{WebProto~\cite{DBLP:conf/saso/UsbeckB13}}: a pioneer no longer maintained playground for Proto
\item \lenf{ScalaFiddle \& Scastie}: well-known Scala playgrounds
\end{itemize}

\end{block}
\end{frame}

\begin{frame}[fragile]{Goals}

\begin{itemize}
\item Reducing the learning time of \scafi{}
\item Inject different aggregate programming languages
\item Zero-install distribution
\item Support also remote program execution
\item Reuse as much as possible the \scafi{} codeline
\end{itemize}
  
\end{frame}

\section{Contribution}

\begin{frame}[fragile]{Architecture}

\begin{itemize}
\item A language-agnostic web interface to visualize \lenf{aggregate} systems
\item Support abstraction language and system dependent
\item Local \scafi{} simulator 
\item Remote Scala code compilation
\end{itemize}
\imgh{0.9}{architecture}
\end{frame}

\begin{frame}[fragile]{Current Usage Scenario}


\begin{block}{Learning and education}
  \begin{itemize}
  \item Exposing an environment with minimal requirements
  \item Sequence of guided example
  \item Simplified access to the simulated sensors and actuators
  \end{itemize}
\end{block}

\begin{block}{Fast Prototyping}
  \begin{itemize}
  \item Aggregate programming require an iterative, incremental process of program refinement
  \item Our approach supports this workflow proving a simulated enviroment with zero instalation overhead
  \end{itemize}
  
\end{block}

\end{frame}

\begin{frame}[fragile]{Demo}
  \centering
  \url{https://scafi.github.io/web/}
\imgh{0.8}{scafi-web-large.png}

\end{frame}
  
%%Todo
\section{Conclusion \& Future Work}

\begin{frame}{Conclusion}

%\imgh{0.9}{example-image}
%

\end{frame}

\begin{frame}[allowframebreaks]{Bibliography}
\def\bibfont{\footnotesize}
\printbibliography
\end{frame}

\end{document}


% \sizedrangedcode{\ssmall}{3}{30}{\labdir/code/code_sec1.txt} 
