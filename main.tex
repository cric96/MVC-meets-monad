\documentclass[presentation]{beamer}
\usepackage{notes_style_v3}
\usecolortheme{myct3}
\usepackage[orientation=landscape,size=custom,width=16,height=9.75,scale=0.5,debug]{beamerposter}
\usepackage{multimedia}
\usepackage{tikz}
\usetikzlibrary{arrows,shapes.geometric,positioning,automata}
\usetikzlibrary{trees,decorations.pathmorphing,calc}
\usetikzlibrary{intersections}

\title[MVC Meets Monad]{MVC meets Monads}
\subtitle{A functional-based design for a classic architectural pattern}

\author[G.Aguzzi]
{\textbf{Gianluca Aguzzi}}

\usepackage[backend=bibtex,style=alphabetic]{biblatex} %,citestyle=authoryear

\usepackage{pifont}
\usepackage{marvosym}
\newcommand{\demoSymbol}{{\Large \ding{249}}} %{{\Large \Faxmachine}} %{\ding{43}}%

\institute[shortinst]{\normalsize 
%\inst{1} 
\textsc{Alma Mater Studiorum}--Universit\`a di Bologna, Cesena, Italy 
}

\bibliography{biblio} 
\definecolor{pbfilling}{HTML}{ffffff}% filling color for the progress bar
%\definecolor{pbfilling}{HTML}{aaaaaa}% background color for the progress bar
\definecolor{pbbackground}{rgb}{0.36, 0.54, 0.66}

\makeatletter
\def\progressbar@progressbar{} % the progress bar
\newcount\progressbar@tmpcounta% auxiliary counter
\newcount\progressbar@tmpcountb% auxiliary counter
\newdimen\progressbar@pbht %progressbar height
\newdimen\progressbar@pbwd %progressbar width
\newdimen\progressbar@tmpdim % auxiliary dimension
\progressbar@pbwd=\paperwidth
\progressbar@pbht=0.5ex
% the progress bar
\def\progressbar@progressbar{%
    \progressbar@tmpcounta= \insertframenumber % max = ?
    \progressbar@tmpcountb=\inserttotalframenumber      
    \progressbar@tmpdim=.5\progressbar@pbwd
    \multiply\progressbar@tmpdim by \progressbar@tmpcounta
    \divide\progressbar@tmpdim by \progressbar@tmpcountb
    \progressbar@tmpdim=2\progressbar@tmpdim
  \begin{tikzpicture}[rounded corners=0,very thin]
    \shade[top color=pbbackground,bottom color=pbbackground,middle color=pbbackground]
      (0pt, 0pt) rectangle ++ (\progressbar@pbwd, \progressbar@pbht);
      \shade[draw=pbfilling,top color=pbfilling,bottom color=pbfilling,middle color=pbfilling] %
        (0pt, 0pt) rectangle ++ (\progressbar@tmpdim, \progressbar@pbht);
  \end{tikzpicture}%
}

\setbeamertemplate{headline}{%
\leavevmode%
  \vbox{
  \hbox{
    \begin{beamercolorbox}[wd=1\paperwidth,ht=0.7ex,sep=0pt,center,dp=0ex]{white}%
    \progressbar@progressbar%
    \end{beamercolorbox}%
  }
  \hbox{%
  \hypersetup{linkcolor=white}
    \begin{beamercolorbox}[wd=1.02\paperwidth,ht=2.5ex,dp=1.125ex]{headline}%
    \insertsubsectionnavigationhorizontal{\paperwidth}{\bfseries\hfill}{\hfill}
    \end{beamercolorbox}%
  }
  }
}

\let\oldcite\cite
%%% additional documents commands
\newcommand{\scafiweb}{{\sc{}ScaFi-Web}}
\newcommand{\scafi}{{\sc{}ScaFi}}
\renewcommand{\cite}[1]{{\color{blue}\oldcite{#1}}}

\begin{document}

\frame[label=coverpage,noframenumbering,plain]{
\titlepage

  \begin{center}
  Talk {@}\\
  \textbf{PPS 2021} \\
  \end{center}
}

\section{It is all about composition}
\begin{frame}[fragile]{It is all about composition \cite{milewski2019category}}
\begin{itemize}
\item Humans tend to divide complex problems in multiple pieces
\item Then, solve each piece
\item Finally compose all solutions to solve the initial problem
\item Functional programming is a good way to compose different solutions :)
\end{itemize}

%\imgv{0.3}{collective-comm.png}

\section{Lecture goals}
\end{frame}
\begin{frame}[fragile]{Lecture goals}
\begin{itemize}
\item Show an end-to-end \emph{functional} application
\item Leverage some well-consolidated functional libraries
\item Understand limitations (if any) and the improvements
\end{itemize}
\end{frame}

\section{Target application}
\begin{frame}[fragile]{Target application: Tic Tac Toe}
\imgh{0.4}{tictactoe.png}
\begin{center}
  Repository: \url{https://github.com/cric96/scala-functional-gui}
\end{center}
\end{frame}

\section{OOP Design}
\begin{frame}[fragile]{OOP Design}
\begin{block}{Everything is an object}
  \begin{itemize}
  \item Clear interface
  \item Incapsulated state
  \item Side effect as methods call
  \end{itemize}
\end{block}

\end{frame}

\section{Model}
\begin{frame}[fragile]{Model}
\begin{block}{Core logic}
  \begin{itemize}
    \item Main data to describe model entities (e.g. players?)
    \item Main methods to describe the game logic
  \end{itemize}
\end{block}
\begin{lstlisting}[language=Java]
/* two players (X, O)  */
enum Player {
    X, O, None; //Or? Optional?
}
/* a board 3x3 */
interface TicTacToe {
    Player get(int X, int Y);
    TicTacToe (or void??) update(int x, int y, Player p);
    boolean isOver;
    Player getTurn;
}
\end{lstlisting}
\end{frame}

\section{View}
\begin{frame}[fragile]{View}
\begin{block}{Representation and IO boundary}
  \begin{itemize}
    \item Describes what type of data it can consume (for rendering porpuse)
    \item Catches how to handle user's inputs
    \item Ideally, the View could be totally decoupled from the Model
  \end{itemize}
\end{block}
\begin{lstlisting}[language=Java]

//a la' view model
interface ViewBoard {
    List<String> getRow(int row);
    List<List<String>> getAllBoard();
}
interface View extends ClickCellSource {
    void render(ViewBoard board);
    void winner(String player);
}
\end{lstlisting}
\end{frame}

\begin{frame}[fragile]{View}
\begin{block}{Desing pattern}
  \begin{itemize}
    \item the Input handler is usually solved exploiting the Observer \cite{gamma1995design} pattern (also called Listener). 
  \end{itemize}
\end{block}
\begin{lstlisting}[language=Java]
public interface ClickCellSource {
  void attach(Observer observer);
  interface Observer {
      void notify(int X, int Y);
  }
}
\end{lstlisting}
\end{frame}

\section{Controller}
\begin{frame}[fragile]{Controller}
\begin{block}{Goals}
  \begin{itemize}
    \item Coordinates the interation between View and Model worlds
    \item Handles concurrency
    \item Adapts data 
  \end{itemize}
\end{block}

\begin{lstlisting}[language=Java]
public interface Game extends ClickCellSource.Observer {
  void start();
}

public class TicTacToeGame implements Game {
  private final TicTacToe ticTacToe;
  private final TicTacToeView ticTacToeView;

  public static TicTacToeGame playWith(
      final TicTacToe ticTacToe, 
      final TicTacToeView ticTacToeView) {...}
  ....

\end{lstlisting}
\end{frame}

\section{Putting all together}
\begin{frame}[fragile]{Putting all together}

\begin{lstlisting}[language=Java]
public static void main(String[] args) {
  final TicTacToeView view = SwingView.createAndShow(800, 600);
  final TicTacToe model = TicTacToeFactory.startX();
  final Game game = TicTacToeGame.playWith(model, view);
  game.start();
}
\end{lstlisting}
\begin{block}{Clean enough, isn't it?}
  What do you think?
  \begin{itemize}
    \item Try to rethink the design phase using "functional" abstractions
    \item (Monads? Functions? Algebraic Data Type?)
    \item Let's go to the "functional" side :)
  \end{itemize}
\end{block}
\end{frame}
%%Todo

\section{Libraries}
\begin{frame}[fragile]{Libraries}

\begin{block}{\href{https://typelevel.org/cats/}{Cats} \cite{scalacats2017}}
  provides abstractions for functional programming in the Scala programming language.
\end{block}

\begin{block}{\href{https://monix.io/}{Monix}}
  high-performance Scala / Scala.js library for composing asynchronous, event-based programs.
\end{block}

\end{frame}
%%Todo
\section{Task}
\begin{frame}[fragile]{Task}

\begin{block}{Definition}
  Task represents a specification for a possibly lazy or asynchronous computation, which when executed will produce a data as a result, along with possible side-effects.
\end{block}
\begin{lstlisting}[language=Scala]
trait Task[+A] {
  final def flatMap[B](f: A => Task[B]): Task[B] = ...
  final def map[B](f : A => B): Task[B] = ...
  //some interesting extesions
  def memoize: Task[A] = ...
}
object Task {
    def pure[A](a : A) : Task[A]
    def defer[A](a : Task[A]) : Task[A]
}
\end{lstlisting}
\begin{center}
  What does it refer you to?
\end{center}

\end{frame}

\section{Task}
\begin{frame}[fragile]{A Little task}

\begin{lstlisting}[language=Scala]
val scheduler = monix.execution.Scheduler.Implicits.global

def someComputation(data : Long) : Task[Long] = 
    Task.pure(data * 1000)

def log(value : String) : Task[Unit] = Task { println(value) }

val main = for {
  data <- someComputation(4)
  _ <- log(s"computations ends with value $data")
} yield (data)

main.runToFuture(scheduler)
\end{lstlisting}
\begin{center}
  \href{https://scalafiddle.io/sf/C4Qon6a/1}{Fiddle}
\end{center}

\end{frame}

\section{Observable}
\begin{frame}[fragile]{Observable}

\begin{block}{Definition}
A data type for modelling and processing asynchronous and reactive streaming of events with non-blocking back-pressure.
\end{block}

\begin{block}{Functional Reactive Programming \cite{wiki:functional} \cite{blackheath2016functional}}
  \begin{itemize}
    \item The program is expressed as a reaction to its inputs, or as a flow of data.
    \item We can use Observable to implement it
    \item Out of this talk, if you are interested, Conal Elliot (\url{http://conal.net/}) proposes a lot of materials about this topic.
  \end{itemize}
\end{block}

\end{frame}

\section{Observable}
\begin{frame}[fragile]{An example with Scala.js :)}
\begin{lstlisting}[language=Scala]
val textInput = input(placeholder := "write text here").render
//unsafe "boundary"
val subject = PublishSubject[String]() 
textInput.oninput = _ => subject.onNext(textInput.value)
val result = p.render
Fiddle.print(div(textInput), result)
//safe part
val inputStream = subject.share //API of the model
val computation = for {
  text <- inputStream
  _ <- Observable.pure(result.innerText = text)
} yield()

computation.foreachL(a => a).runToFuture
\end{lstlisting}  
\begin{center}
  \href{https://scalafiddle.io/sf/0uDr1Cr/1}{Fiddle}
\end{center}

\end{frame}

\section{MVC + Functional}
\begin{frame}[fragile]{MVC + Functional, Intuition}
\begin{itemize}
  \item Model is described using Algebraic Data Type and operations (in a separated module)
  \item View is modelled using monadic abstraction to wrap side effects
  \item Controller coordinates model and view without introducing other side effects and using a monadic manipulation (e.g. via flatmap)
\end{itemize}


\begin{block}{Applied in our case}
\begin{itemize}
  \item Observable describes the user's inputs flow
  \item Other computations (e.g. lazy and with side effects) are wrapped with Task
  \item Let's start again :) (using Scala)
\end{itemize}
\end{block}

\end{frame}

\section{MVC + Functional}
\begin{frame}[fragile]{Model}

\begin{block}{Domain Data}
\begin{itemize}
  \item Immutable data structures, operations can be described in a separated module
  \item We can enhance ADT with type classes
\end{itemize}
\end{block}

\begin{lstlisting}[language=scala]
sealed trait TicTacToe {
  def board: Map[Position, Player]
}
object TicTacToe {
  val defaultSize: Int = 3
  type Position = (Int, Int) //type aliases improve readability
  case class InProgress(turn: Player, board: Map[Position, Player]) extends TicTacToe
  case class End(winner: Player, board: Map[Position, Player]) extends TicTacToe
  sealed trait Player {
    def other: Player = ...
  }
  case object X extends Player
  case object O extends Player
}
\end{lstlisting}

\end{frame}

\section{MVC + Functional}
\begin{frame}[fragile]{Model}

\begin{block}{Operations}
\begin{itemize}
  \item Behaviour is described in terms of a function that evolves into a TicTacToe instance
  \item Other ideas? State monad?
\end{itemize}
\end{block}

\begin{lstlisting}[language=scala]
object TicTacToeOps {
  def advanceWith(ticTacToe: TicTacToe, hit: Position): TicTacToe = {
    val updateGame = for {
      _ <- rightPosition(ticTacToe, hit)
    } yield updateBoard(ticTacToe, hit)
    updateGame.getOrElse(ticTacToe)
  }
  def rightPosition(
    ticTacToe: TicTacToe, 
    position: Position): Option[TicTacToe] = ....
  def updateBoard(
    ticTacToe: TicTacToe, 
    position: Position): TicTacToe = ...
  ...
}
\end{lstlisting}

\end{frame}

\section{MVC + Functional}
\begin{frame}[fragile]{View}

\begin{block}{Input-Output management}
\begin{itemize}
  \item IO is something outside of the model
  \item Boundary wraps something \emph{outside} the control of the domain logic (e.g. GUI, Console, Socket) 
  \item Boundary, conceptually, can be defined as independent from a specific model data (and input data)
  \item There are any problems with this solution?
\end{itemize}
\end{block}

\begin{lstlisting}[language=scala]
trait Boundary[-Model, +Input] { //why generic?
  def init(): Task[Unit] = Task.pure {}

  def input: Observable[Input]

  def consume(model: Model): Task[Unit]
}
\end{lstlisting}

\end{frame}


\section{MVC + Functional}
\begin{frame}[fragile]{Controller}

\begin{block}{Application strucutres}
  \begin{itemize}
    \item The application flow could be either \emph{reactive} or \emph{proactive}
    \item When it is reactive, the Controller updates the Model only when a new input occurs
    \item When it is proactive, the Controller updates the Model independently from the input flow
  \end{itemize}
\end{block}

\begin{block}{Intuition}
\begin{itemize}
  \item Controller is a module (object) that exposes two function, i.e. reactive and proactive
  \item Both functions, return a Task (i.e. wrap a computation), abstracting over the typology
\end{itemize}
\end{block}
\begin{block}{Can we build it independently from Model and View?}
  \begin{itemize}
    \item We only need a function accepting an Input (I) and evolving the Model (M) returning a Task[M] (why?).
    \item An initial Model
    \item A boundary used for our application
  \end{itemize}
\end{block}

\end{frame}


\section{MVC + Functional}
\begin{frame}[fragile]{Controller, Structure}
\begin{lstlisting}[language=Scala]
type UpdateFn[Model, Input] = (Model, Long, Seq[Input]) => Task[Model]

object Controller {
  
  case class ProactiveConfig(timeTarget: FiniteDuration = 33.millis, bufferSize: Int = 5)

  def reactive[M, I](
    boundary: Boundary[M, I], 
    model: M, 
    updateLogic: UpdateFn[M, I]): Task[Unit] = ....

  def proactive[M, I](
      boundary: Boundary[M, I],
      model: M,
      updateLogic: UpdateFn[M, I],
      config: ProactiveConfig
  ): Task[Unit] = ...

}
\end{lstlisting}
\end{frame}


\section{MVC + Functional}
\begin{frame}[fragile]{View, Implementation}

\begin{lstlisting}[language=Scala]
  class View(width: Int = 800, height: Int = 600) extends Boundary[TicTacToe, Hit] {
    ....
    override def input: Observable[Hit] = Observable
      .fromTask(cells)
      .flatMapIterable(a => a)
      .map(checkObservable)
      .merge
    override def consume(model: TicTacToe): Task[Unit] = {
      for {
        frame <- container.asyncBoundary(swingScheduler)
        panel <- board
        _ <- io(panel.removeAll())
        _ <- renderButtons(panel, model.board)
        _ <- renderVictory(model)
        _ <- io(frame.repaint()) //force repaint
        _ <- io(frame.setVisible(true)) //force repaint
      } yield ()
    }
    ...
  }
\end{lstlisting}
\end{frame}

\section{MVC + Functional}
\begin{frame}[fragile]{Application}

\begin{lstlisting}[language=Scala]
object TicTacToeSafe extends TaskApp {

  override def run(args: List[String]): Task[ExitCode] = {
    val view: Boundary[TicTacToe, Hit] = new View()
    //main logic
    val gameLogic: UpdateFn[TicTacToe, Hit] = UpdateFn.timeIndependent { 
      case (world, inputs) => TicTacToeInputProcess(world, inputs)
    }
    val loop = Controller.reactive(view, InProgress(X, Map.empty), gameLogic)
    loop.map(_ => ExitCode.Success)
  }
}
\end{lstlisting}
\end{frame}

\section{Discussion}
\begin{frame}[fragile]{Discussion}

\begin{itemize}
  \item My effort consists in merging MVC concepts with monads/functional programming
  \item Have you found some "impurities"? Or some lacks?
  \item Do you prefer the OOP solution or the "Functional" solution? And why?
\end{itemize}

\begin{block}{Possible lacks (you can try to solve them as exercise :D)}
\begin{itemize}
  \item Error handling? 
  \item Concurrency fine-control? Parallelism?
  \item Pure "monadic" view (my interface exposes side effects, isn't it?)
  \item Cleaner Controller (monad transform?)
  \item Other application domain
  \item Other GUI supports
\end{itemize}
\end{block}
\end{frame}


\begin{frame}[allowframebreaks]{Bibliography}
\def\bibfont{\footnotesize}
\printbibliography
\end{frame}

\end{document}


% \sizedrangedcode{\ssmall}{3}{30}{\labdir/code/code_sec1.txt} 
